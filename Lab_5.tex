%%%%%%%%%%%%%%%%%%%%%%%%%%%%%%%%%%%%%%%%%%%%%%%%%%%%%%%%%%%%%%%%%%%%%
%Adriana Oliveira
%ECE 351-51
%Lab: 5
%Due date: 10/08/2019
%notes: 
%%%%%%%%%%%%%%%%%%%%%%%%%%%%%%%%%%%%%%%%%%%%%%%%%%%%%%%%%%%%%%%%%%%%%%

\documentclass[12pt]{report}
\usepackage[a4paper]{geometry}
\usepackage[myheadings]{fullpage}
\usepackage{fancyhdr}
\usepackage{lastpage}
\usepackage{graphicx, wrapfig, subcaption, setspace, booktabs}
\usepackage[T1]{fontenc}
\usepackage[font=small, labelfont=bf]{caption}


\usepackage{fourier}
\usepackage[protrusion=true, expansion=true]{microtype}
\usepackage[english]{babel}
\usepackage{sectsty}
\usepackage{url, lipsum}
\usepackage{indentfirst}
\usepackage{listings}

\newcommand{\HRule}[1]{\rule{\linewidth}{#1}}
\onehalfspacing
\setcounter{tocdepth}{5}
\setcounter{secnumdepth}{5}

%-------------------------------------------------------------------------------
% HEADER & FOOTER
%-------------------------------------------------------------------------------
\pagestyle{fancy}
\fancyhf{}
\setlength\headheight{15pt}
\fancyhead[L]{Adriana Oliveira}
\fancyhead[R]{ECE351}
\fancyfoot[R]{Page \thepage\ of \pageref{LastPage}}
%-------------------------------------------------------------------------------
% TITLE PAGE
%-------------------------------------------------------------------------------

\begin{document}

\title{ 
		\\ [2.0cm]
		\HRule{0.5pt} \\
		\LARGE \textbf{\uppercase{ECE351-51: Lab 5}} 
		\HRule{2pt} \\ [0.5cm]
		\normalsize\textsc{October 8, 2019} \vspace*{5\baselineskip}}

\date{}
\author{\normalsize
		Submitted By: \\
		\normalsize Adriana Oliveira\\}

\maketitle

\newpage

\sectionfont{\scshape}
%%%%%%%%%%%%%%%%%%%%%%%%%%%%%%%%%%% Introduction %%%%%%%%%%%%%%%%%%%%%%%%
\section*{Introduction}
\hrule
\vspace{1cm}
\setlength{\parindent}{5ex}
Before we conducted this lab, we solved for the transfer function and impulse response for the following circuit
\begin{figure}[htp]
    \centering
    \includegraphics[width=10cm]{prelab.png}
    \caption{RLC bandpass filter}
    \label{fig:ghj}
\end{figure}

Our goal is to verify the result in the time domain using two different python functions.
%%%%%%%%%%%%%%%%%%%%%%%%%%%%%%%%%%% Methodology%%%%%%%%%%%%%%%%%%%%%%%%%%%%%
\section*{Methodology}
\hrule
\vspace{1cm}
\setlength{\parindent}{5ex}
We started by reducing the RLC circuit to the form below.
\begin{figure}[htp]
    \centering
    \includegraphics[width=10cm]{revised.png}
    \caption{RLC bandpass filter in Laplace domain}
    \label{fig:ghj}
\end{figure}

We then simplified our parallel inductor and capacitor elements and applied voltage division to solve for $v_{out}$. This results in $H(s) = (s/RC)/(s^{2}+(s/RC)+(1/LC)) = (s+2)/(s^{2}+3*s+8)$. Using the sine method, we converted the transfer function to an impulse response in the time domain. This process is available in the equation portion below. 

%%%%%%%%%%%%%%%%%%%%%%% PART 1
\vspace{-0.5cm}
\subsection*{Part 1: Plot h(t)}
\setlength{\parindent}{5ex}
Fist we plot the impulse response using our hand solved impulse. This was done by creating a function for our step response and nesting it within an impulse response function. This function was then plotted over time. \par

\begin{lstlisting}[language=Python, caption=Functions for hand calculations]
#Hand calculated h(t)
steps = 1e-5
t = np.arange(0,1.2e-3+steps,steps)

def u(t):
    if t < 0:
        return 0
    if t >= 0:
        return 1

def h(t):
    y = np.zeros((len(t), 1))
    for i in range(len(t)):
        y[i] = 10356*np.exp(-(5000*(t[i])))*np.sin(18584*(t[i])+(105*np.pi/180))*u(t[i])
    return y
    y = h(t)
\end{lstlisting}
 
 \vspace{5mm}
 Next, we used a the scipy.signal.impulse() function to convert our transfer function. This impulse was also plotted. 
 
 \begin{lstlisting}[language=Python, caption=Python function to convert to t domain]
R = 1000
L = .027
C = 100e-9
X = 1/(R*C)
Y = 1/(L*C)


steps = 1e-5
t = np.arange(0,1.2e-3+steps,steps)

#Python h(t)  
num = [X,0]
den = [1,X,Y]
tout , yout = sig.impulse (( num , den ) , T = t )

\end{lstlisting}
 \vspace{5mm}
The resulting plots represent h(t) for 0< t < 1.2 ms.

\begin{figure}[htp]
    \centering
    \includegraphics[width=15cm]{two_plots.png}
    \caption{Plot for h(t)}
    \label{fig:ghj}
\end{figure}


%%%%%%%%%%%%%%%%%%%%%%% PART 2
\vspace{-\baselineskip}
\vspace{-0.5cm}
\subsection*{Part 2: Plotting h(t)*u(t)}
\setlength{\parindent}{5ex}

Next, we plotted the step response with the transfer function. This involved finding the step response using scipy.signal.step().

 \begin{lstlisting}[language=Python, caption=Python function to convert to t domain]

\end{lstlisting}
#h(t)*u(t) 
tout , yout = sig.step (( num , den ) , T = t )

\vspace{1cm}
This results in the following plot of h(t)*u(t) for 0< t < 1.2 ms.

\begin{figure}[htp]
    \centering
    \includegraphics[width=15cm]{last.png}
    \caption{Plot for h(t)*u(t)}
    \label{fig:ghj}
\end{figure}
 %%%%%%%%%%%%%%%%%%%%%%%%%%%%%%%%%%% Equations %%%%%%%%%%%%%%%%%%%%%%%%%%%%%%%%%%%
\section*{Equations}
\hrule
\vspace{1cm}
\setlength{\parindent}{5ex}

\begin{center}
    \textbf{Sine method}
\end{center}

$$h(t) = (|g|/w)*{e^{at}}*sin(wt+{g_{angle}})$$

$$ P = (-(1/RC)+((1/RC)^{2}-4*(1/LC))^(1/2))/2$$

$$g = (1/RC)|_{s=P}$$

$$h(t) = 10356*e^{-5000t}*sin(18584*t + 105^{o})$$

 \newpage
%%%%%%%%%%%%%%%%%%%%%%%%%%%%%%%%%%% Conclusion %%%%%%%%%%%%%%%%%%%%%%%%%%%%%%%%%%%

\section*{Conclusion}
\hrule
\vspace{1cm}
\setlength{\parindent}{5ex}
The plots for h(t) are the same as h(t)*u(t) except the convolution with a step function begins at zero. 

%%%%%%%%%%%%%%%%%%%%%%%%%%%%%%%%%% Questions %%%%%%%%%%%%%%%%%%%%%%%%%%%%%%%%%%%

\section*{Questions}
\hrule
\vspace{1cm}
\setlength{\parindent}{5ex}
Explain the result of the Final Value Theorem from Part 2 Task 2 in terms of the physical circuit components.

The finite value theorem for this circuit equals zero. This is seen as the amplitude of the wave tapers off on the x-axis. As a result, the circuit will filter out the incoming frequency and return to a steady state. 

%%%%%%%%%%%%%%%%%%%%%%%%%%%%%%%%%%% Appendix %%%%%%%%%%%%%%%%%%%%%%%%%%%%%%%%%%%
\newpage
\section*{Appendix}
\hrule
\vspace{1cm}
\setlength{\parindent}{5ex}

\begin{lstlisting}[language=Python, caption= Part 1: Task 1]
# -*- coding: utf-8 -*-
"""
Spyder Editor

This is a temporary script file.
"""
import numpy as np
import matplotlib.pyplot as plt
import scipy.signal as sig
import control
import pandas

#Part 1: Plot impulse response of circuit

#Hand calculated h(t)
def u(t):
    if t < 0:
        return 0
    if t >= 0:
        return 1

def h(t):
    y = np.zeros((len(t), 1))
    for i in range(len(t)):
        y[i] = 10356*np.exp(-(5000*(t[i])))*np.sin(18584*(t[i])+(105*np.pi/180))*u(t[i])
    return y
    y = h(t)

y = h(t)

#Python h(t)  
R = 1000
L = .027
C = 100e-9
X = 1/(R*C)
Y = 1/(L*C)

steps = 1e-5
t = np.arange(0,1.2e-3+steps,steps)


num = [X,0]
den = [1,X,Y]
tout , yout = sig.impulse (( num , den ) , T = t )


myFigSize = (10,8)
plt.figure(figsize=myFigSize)

#Graph for hand calc h(t) 
myFigSize = (10,8)
plt.figure(figsize=myFigSize)
plt.subplot(1,2,1)
plt.plot(t,y)
plt.grid(True)
plt.xlabel('time')
plt.ylabel('h(t)')
plt.title('Hand calculated h(t)')
plt.show()

#Graph for python h(t) 
plt.subplot(1,2,2)
plt.plot(t,yout)
plt.grid(True)
plt.xlabel('time')
plt.ylabel('h(t)')
plt.title('Python calculated h(t)')


#Graph for h(t)*u(t) 
tout , yout = sig.step (( num , den ) , T = t )

myFigSize = (10,8)
plt.figure(figsize=myFigSize)
plt.subplot(2,1,1)
plt.plot(t,yout)
plt.grid(True)
plt.xlabel('time')
plt.ylabel('h(t)*u(t)')
plt.title('Python calculated h(t)*u(t)')
plt.show()

#Final Value Theorem 
#s = 0
#FVT = s*((s+2)/(s^{2}+3*s+8))
FVT = 0

\end{lstlisting}

\end{document}
